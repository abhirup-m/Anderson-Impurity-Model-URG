\subsection{Without spin-spin interaction}
The model is the usual single-impurity Anderson model Hamiltonian.
\beq
\ham = \sum_{k\sigma}\epsilon_k \hat n_{k\sigma} + \sum_{k\sigma} \rr{V_{k} c^\dagger_{k\sigma} c_{d\sigma} + h.c.} + \epsilon_{d}\sum_\sigma  \hat n_{d\sigma} +  U \hat n_{d\ua} \hat n_{d\da}
\eeq
To allow the calculation of both particle and hole kinetic energies, we will write the kinetic energy part as \(\sum_{k\sigma}\epsilon_k \tau_{k\sigma}\) and drop the extra constant part.
\pb At first order, the rotated Hamiltonian is
\beq[newh]
\ham_{j-1} = \ham_0 + \sum_{q\beta}\tau_{q\beta}\cc{c^\dagger_{q\beta}\text{Tr}_{q\beta}\rr{\ham c_{q\beta}},\eta_{q\beta}}
\eeq
\(\ham_0\) is the part of the Hamiltonian that conserves the operator \(\hat n_{q\beta}\).
\beq[term1]
\ham_0 = \sum_{k\sigma}\epsilon_k \hat n_{k\sigma} + \sum_{k<\Lambda_N,\sigma} \rr{V_{k} c^\dagger_{k\sigma} c_{d\sigma} + h.c.} + \epsilon_{d}\sum_\sigma  \hat n_{d\sigma} +  U \hat n_{d\ua} \hat n_{d\da}
\eeq
We take the full Hamiltonian as our \il{\ham_j}.
Since this is the first step of the RG, the shell being decoupled is the highest one, which we call \il{\Lambda_N}.

\subsubsection*{Particle Sector}
The particle sector involves only particle excitations. The state \(q\beta\) is occupied in the intermediate (excited) state. This means that the state must be vacant in the initial state, so we must have \(\epsilon_q > 0\). This contribution will be given by the first term in the anticommutator of eq.~\ref{newh}.
From the prescription (eq.~\ref{deltap}), the renormalization in the particle sector is
\beq
\Delta^+ \ham = \sum_{q\beta} \text{Tr}\rr{c^\dagger_{q\beta}\ham }c_{q\beta}\fr{1}{\omega_e - \ham^D}c_{q\beta}^\dagger \text{Tr}\rr{\ham c_{q\beta}}\\
\eeq
We now compute each of the terms.
\beq
\text{Tr}_{q\beta}\rr{\ham c_{q\beta}} &= \sum_{k\sigma} V_k \text{Tr}_{q\beta}\rr{c^\dagger_{k\sigma}c_{d\sigma}c_{q\beta}}\\
				       &= \sum_{k\sigma} V_k c_{d\sigma}\delta_{\sigma\beta}\delta_{kq}\\
				       &= V_q c_{d\beta}\\
\text{Tr}_{q\beta}\rr{c_{q\beta}^\dagger \ham } &= V^*_q c^\dagger_{d\beta}
\eeq
\beq[dham]
\ham^D &= \sum_{k\sigma}\epsilon_k \tau_{k\sigma} + \epsilon_{d}\sum_\sigma  \hat n_{d\sigma} +  U \hat n_{d\ua} \hat n_{d\da}
\eeq
\beq
\text{Tr}_{q\beta}\rr{\ham^D \hat n_{q\beta}} &= \sum_{k<\Lambda_N,\sigma}\epsilon_k \tau_{k\sigma} + \epsilon_{d}\sum_\sigma  \hat n_{d\sigma} +  U \hat n_{d\ua} \hat n_{d\da} + \hf\epsilon_q
\eeq
There is a more straightforward way of getting these expressions. Some thought reveals that \(c_{q\beta}^\dagger\text{Tr}_{q\beta}\rr{\ham c_{q\beta}}\) is, by definition, the part of the Hamiltonian that scatters from electrons \textit{not at} \(q\beta\) to \(q\beta\). In other words,\textbf{ it is that off-diagonal part of the Hamiltonian that involves a \(c_{q\beta}^\dagger\)}. That part is, of course, \(V_q c^\dagger_{q\beta}c_{d\beta}\). Similarly, \(\text{Tr}_{q\beta}\rr{c^\dagger_{q\beta}\ham}c_{q\beta}\) is the off-diagonal part that has a \(c_{q\beta}\), \(V_q^* c^\dagger_{d\beta}c_{q\beta}\). Finally, the term in the denominator of \(\eta\) is simply the diagonal part of the Hamiltonian, which in our case is the kinetic energies of all the electrons and the impurity diagonal part. The point of this paragraph is that one can write down these terms simply by looking at the Hamiltonian and without carrying out any trace.
\beq
\eta_{q\beta} &\equiv \text{Tr}_{q\beta}\rr{c^\dagger_{q\beta}\ham }c_{q\beta}\fr{1}{\omega_e - \ham^D}\\
	      &= \text{Tr}_{q\beta}\rr{c^\dagger_{q\beta}\ham}c_{q\beta}\fr{1}{\omega_e - \text{Tr}_{q\beta}\rr{\ham^D \hat n_{q\beta}}\hat n_{q\beta}}\\
	      &= V^*_q c^\dagger_{d\beta}c_{q\beta}\fr{1}{\omega_e - \rr{\sum_{k<\Lambda_N,\sigma}\epsilon_k \tau_{k\sigma} + \epsilon_{d}\sum_\sigma  \hat n_{d\sigma} +  U \hat n_{d\ua} \hat n_{d\da} + \hf\epsilon_q}\hat n_{q\beta}}\\
	      &= V^*_q c^\dagger_{d\beta}c_{q\beta}\fr{1}{\omega_e - \rr{\epsilon_{d}\sum_\sigma \hat n_{d\sigma} + U \hat n_{d\ua}\hat n_{d\da} + \hf\epsilon_q}}\\
\eeq
At the last step, I dropped the lower shell electrons from the denominator to simplify the calculation. I also replaced \(\hat n_{q\beta}\) with 1 because the propagator has a \(c_{q\beta}\) to the left.
\pb Note that since this term has a \il{c^\dagger_{d\beta}}, it will survive only when acting on a state with \il{\hat n_{d\beta} = 0}.
Hence we can drop the terms \il{\hat n_{d\ua}\hat n_{d\da}} and \il{\epsilon_{d\beta}\hat n_{d\beta}} in the denominator.
Putting together the individual pieces, we can now write the whole thing:
\beq
\Delta^+ \ham &= \sum_{q\beta}\eta_{q\beta}c^\dagger_{q\beta}\text{Tr}_{q\beta}\rr{\ham c_{q\beta}}\\
	    &= \sum_{q\beta}V^*_q c^\dagger_{d\beta}c_{q\beta}\fr{1}{\omega_e - \hf\epsilon_q - \epsilon_{d}\hat n_{d\ol\beta}}V_q c^\dagger_{q\beta}c_{d\beta} \\
	    &= \sum_{q\beta}|V_q|^2 \hat n_{d\beta}\rr{1 - \hat n_{q\beta}}\fr{1}{\omega_e - \hf\epsilon_q - \epsilon_{d}\hat n_{d\ol\beta}}\\
	    &= \sum_{q\beta}|V_q|^2 \hat n_{d\beta}\qq{\fr{\hat n_{d\ol\beta}}{\omega_e - \hf\epsilon_q - \epsilon_{d}} + \fr{1 -\hat n_{d\ol\beta}}{\omega_e - \hf\epsilon_q }}
\eeq
In the last step, we replaced \(\hat n_{q\beta}=0\) because we are working with \(\epsilon_q > 0\). Also, we used 
\beq
\fr{1}{\omega_e - \hf\epsilon_q - \epsilon_{d}\hat n_{d\ol\beta}} &= \fr{\hat n_{d\ol\beta}}{\omega_e - \hf\epsilon_q - \epsilon_{d}} + \fr{1 -\hat n_{d\ol\beta}}{\omega_e - \hf\epsilon_q }\\
\eeq
From the expression of \(\Delta^+ \ham\), we see two terms: the first term renormalizes the energy of the doubly-occupied state \(E_2\), while the other term renormalizes the singly-occupied state \(E_1\):
\beq
\Delta^+ E_1 &= \sum_{q,\epsilon_q>0}\fr{|V_q|^2}{\omega_e - \hf \epsilon_q}\\
\Delta^+ E_2 &= 2\sum_{q,\epsilon_q>0}\fr{|V_q|^2}{\omega_e - \hf \epsilon_q - \epsilon_d}
\eeq
\subsubsection*{Hole Sector}
The hole sector consists of those excitations where the state is occupied in the initial state but vacant in the final state (\(\epsilon_q < 0\)). Again from the prescription, we have
\beq
\Delta^- \ham &= \sum_{q\beta} c_{q\beta}^\dagger \text{Tr}\rr{\ham c_{q\beta}}\eta_{q\beta}\\
	      &= \sum_{q\beta} c_{q\beta}^\dagger \text{Tr}\rr{\ham c_{q\beta}}\fr{1}{\omega_h - \ham^D}\text{Tr}\rr{c^\dagger_{q\beta}\ham }c_{q\beta}\\
\eeq
In this sector, the diagonal part \(\ham^D\) is obtained by changing eq.~\ref{dham}: \(\epsilon_q \to -\epsilon_q\) and \(\tau_{q\beta} = -\hf\):
\beq
\ham^D = \hf\epsilon_q + \epsilon_d\sum_\sigma\hat n_{d\sigma} + U\hat n_{d\ua}\hat n_{d\da}
\eeq
Therefore,
\beq
\Delta^- \ham &= \sum_{q\beta}V_q c^\dagger_{q\beta}c_{d\beta} \fr{1}{\omega_h - \hf\epsilon_q - \sum_{\sigma}\epsilon_{d}\hat n_{d\sigma} - U\hat n_{d\ua}\hat n_{d\da}}V^*_q c^\dagger_{d\beta}c_{q\beta}\\
	      &= \sum_{q\beta}|V_q|^2 c^\dagger_{q\beta}c_{d\beta} \fr{1}{\omega_h - \hf\epsilon_q - \epsilon_d - \rr{\epsilon_d + U}\hat \hat n_{d\ol\beta}}c^\dagger_{d\beta}c_{q\beta}\\
	      &= \sum_{q\beta}|V_q|^2 \rr{1 - \hat n_{d\beta}} \qq{\fr{\hat n_{d\ol\beta}}{\omega_h - \hf\epsilon_q - 2\epsilon_d - U} + \fr{1 - \hat n_{d\ol\beta}}{\omega_h - \hf\epsilon_q - \epsilon_d}}
\eeq
We replaced \(\hat n_{q\beta}=1\) because \(\epsilon_q<0\). The renormalizations in this sector are
\beq
\Delta^- E_0 &= 2\sum_{q,\epsilon_q<0}\fr{|V_q|^2}{\omega_h - \hf \epsilon_q - \epsilon_d}\\
\Delta^- E_1 &= \sum_{q,\epsilon_q<0}\fr{|V_q|^2}{\omega_h - \hf \epsilon_q - 2\epsilon_d - U}
\eeq
\(E_0\) is the energy of the vacant state.
\paragraph{Flow Equations}
The \(\omega\) have to be determined self-consistently or numerically by searching for fixed points where they become identical to the energy eigenvalues. Presently we replace them with the renormalized diagonal part of the initial state (since the \(\omega\) relate to the Hamiltonian where the electron has already been decoupled, it makes sense to compute them from states which do not have these excitations) to make some qualititative analysis. For the particle sector we have then have
\beq[abcd]
\Delta^+ E_1 &= \sum_{q,\epsilon_q>0}\fr{|V_q|^2}{\epsilon_d - \hf\epsilon_q - \hf \epsilon_q} = \sum_{q,\epsilon_q>0}\fr{|V_q|^2}{\epsilon_d - \epsilon_q}\\
\Delta^+ E_2 &= 2\sum_{q,\epsilon_q>0}\fr{|V_q|^2}{2\epsilon_d + U - \hf \epsilon_q- \hf \epsilon_q - \epsilon_d}= 2\sum_{q,\epsilon_q>0}\fr{|V_q|^2}{\epsilon_d + U - \epsilon_q}
\eeq
In the first equation, we replaced \(\omega_e = \epsilon_d - \hf\epsilon_q\), because the impurity must be singly-occupied in the initial state if the term has to renormalize \(E_1\) and the state \(q\beta\) is empty, hence the \(-\hf\epsilon_q\). Similar arguments give
\beq[defg]
\Delta^- E_0 &= 2\sum_{q,\epsilon_q<0}\fr{|V_q|^2}{- \epsilon_q - \epsilon_d}\\
\Delta^- E_1 &= \sum_{q,\epsilon_q<0}\fr{|V_q|^2}{- \hf \epsilon_q - \epsilon_d - U}
\eeq
These results (eqs.~\ref{abcd} and \ref{defg}) are quoted in \cite{hewson}, and give the familiar one-loop RG equations.
\subsubsection{Particle-Hole symmetry}
For a particle-hole symmetric model, we can substitute \(\omega_e = \omega_h = \omega\). This gives
\beq
\Delta^- E_0 &= 2\sum_{q}\fr{|V_q|^2}{\omega - \hf \epsilon_q - \epsilon_d} = \Delta^+ E_2\\
\eeq
This shows that the doublon and holon states remain equidistant from the single-particle level, thus maintaining particle-hole symmetry along the flow.

\subsection{With Kondo-like interaction}
