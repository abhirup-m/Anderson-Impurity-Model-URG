\begin{titlepage}
	\centering
	\rule{\textwidth}{3pt}\\
	\vspace*{20pt}
	{
		\textbf{\LARGE Unitary Renormalization Group Solution}\\[10pt]
		\textbf{\LARGE of The Single-Impurity Anderson Model}\\[10pt]
	}
	\rule{\textwidth}{3pt} % Thick horizontal rule
	\vfill
	{\Large \textbf{MS Project Report}}\\
	\vfill
	\textit{\large{ submitted in partial fulfillment of the requirements for the degree of }}\\
	\vfill
	{\Large \textbf{Master of Science\\}}
	\vfill
	\textit{\large by \\}
	\vfill
	{\Large \textbf{Abhirup Mukherjee \\}}
	\vspace*{5pt}
	{\Large \textbf{(18IP014)\\}}
	\vfill
	\textit{\large under the supervision of \\}
	\vfill
	{\Large \textbf{Dr. Siddhartha Lal}\\
	\vfill
	\today
	\vfill
	\Large Department of Physical Sciences\\
	}
	\vfill
	\includegraphics[scale=0.15]{../figures/logo.png}\\
	\vspace{0.01\textheight}
	\textbf{\Large Indian Institute of Science Education and \\[10pt]}
	\textbf{\Large Research, Kolkata}\
\end{titlepage}
\chapter*{Acknowledgments}
I express my heartfelt gratitude to my supervisor Dr. Siddhartha Lal for providing very useful guidance and very crucial insights into the tough problems. This project would not have been possible without the help of my group senior Siddhartha Patra and former group member Dr. Anirban Mukherjee. Their experience with the method as well as work on similar projects paved the entire journey for me. A special shout out to my friend Mounica Mahankali for all the useful discussions. The support of IISER Kolkata in the form of a junior research fellowship is gratefully acknowledged.

\chapter*{}
\begin{center}
	\vspace*{100pt}
	\Large{\textit{To you, 5 years from now}}
\end{center}

\chapter*{Abstract}
This thesis reports a renormalization group analysis of the single impurity Anderson model (SIAM). The analysis includes a derivation of RG equations for the couplings as well as computation of physical properties. The renormalization group method (URG) is based on unitary transformations that decouple high energy nodes from the Hamiltonian, rendering them integrals of motion. It has been introduced and formalised in refs.~\cite{anirbanurg1,anirbanurg2,anirbanmott1,anirbanmott2}. Some chapters have been devoted to deriving and explaining the method in detail and as well as applying it on some simpler models like the star graph model and the single-channel Kondo model. To give a clearer view of what the URG does, we connect this method to other unitary transformations in the literature, like the Schrieffer-Wolff transformation, poor man's scaling and continuous unitary transformation renormalization group. Having set up the method, we apply it on a generalized version of the SIAM with explicit spin-exchange and charge isospin-exchange couplings. We find strong-coupling fixed points for both the spin and isospin couplings. From the zero mode, we then calculate the ground state wavefunctions, which turnout to be spin singlet and isospin singlet. We thermodynamic quantities like the magnetic susceptibility and the specific heat. We then extract an effective Hamiltonian for the cloud of electrons that screen the impurity. This is done by integrating out the impurity from the fixed point Hamiltonian. This process of integrating out generates interactions among the members of the Kondo cloud. This effective Hamiltonian is found to contain both Fermi liquid as well as four-Fermion off-diagonal interaction terms. We calculate the zero temperature Wilson ratio from the local Fermi liquid formulation of Nozières, which turns out to be 2 for the Kondo regime of the SIAM. We also calculate the change in Luttinger's volume as we move from the high energy fixed point to the low energy fixed point, by tracking the changes in the number of poles of the impurity and conduction bath Greens functions. We find that the total Luttinger volume increases by 1, because the impurity state also gets added to the Fermi volume. We also compute the impurity spectral across the RG flow, and it is seen that the three peak structure at the local moment fixed point evolves into a single peak structure at the strong-coupling fixed point, demonstrating the transfer of spectral weight to zero frequency in the low energy theory. We finally calculate the mutual information and correlations along the RG flow between impurity and a Kondo cloud electron, as well as between two members of the Kondo cloud. Both the measures increase towards the strong-coupling fixed point, showing that the flow towards low energies is accompanied by a substantial increase in the entanglement content.

\listoffigures
\listoftables
\tableofcontents
