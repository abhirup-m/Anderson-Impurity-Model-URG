\documentclass[aspectratio=169]{beamer}
\usepackage{xcolor}
\definecolor{maroon}{rgb}{0.6,0,0}
\definecolor{bottlegreen}{rgb}{0,0.3,0}
\usepackage{lmodern,graphicx}
\usetheme{focus}
\newcommand{\cen}[1]{\begin{center}{#1}\end{center}}
\usepackage[absolute,overlay]{textpos}

\setbeamercolor{framesource}{fg=gray}
\setbeamerfont{framesource}{size=\tiny}

\newcommand{\source}[1]{\begin{textblock*}{4cm}(8.7cm,8.6cm)
    \begin{beamercolorbox}[ht=0.5cm,right]{framesource}
        \usebeamerfont{framesource}\usebeamercolor[fg]{framesource} Source: {#1}
    \end{beamercolorbox}
\end{textblock*}}

\title{Unitary Renormalization Group Approach to the Single-Impurity Anderson model}
\subtitle{}
\date{\today}
\author{Abhirup Mukherjee (18IP014)\\[5mm]{Supervisor: Dr. Siddhartha Lal}}

\institute{IISER Kolkata}
\date{January 8, 2021}

\begin{document}


\begin{frame}
\maketitle
\end{frame}

\begin{frame}{Outline}

\begin{itemize}
  \item The model
	  \vspace*{20pt}
  \item Motivation
	  \vspace*{20pt}
  \item Unitary Renormalization Group (URG) formalism
	  \vspace*{20pt}
  \item Results 
\end{itemize}

\end{frame}

\begin{frame}{The Single-Impurity Anderson Model}

%\begin{figure}
%\centering
%\def\svgwidth{\columnwidth}
%\centering
%\input{model.pdf_tex}
%\end{figure}
%\begin{figure}
%\centering
%\includegraphics[scale=0.55]{model_scheme.png}
%\end{figure}
\only<+>{
\scalebox{1.38}{
\hspace*{-10pt}\(\mathcal{H}_\text{siam} = \sum_{k\sigma}\epsilon_k \hat n_{k\sigma} + \sum_{k\sigma}\left[V(k)c^\dagger_{k\sigma}c_{d\sigma} + \text{h.c.}\right] + \epsilon_d \sum_\sigma \hat n_{d\sigma} + U\hat n_{d\uparrow}\hat n_{d\downarrow}
\)
}
\begin{figure}
\centering
\includegraphics[scale=0.55]{model_scheme.png}
\end{figure}
}
\only<+>{
	\vspace*{-20pt}
\cen{\Large\textbf{Poor Man's Scaling Results}}
\begin{tabular}{cl}  
\begin{tabular}{l}
             \parbox{0.5\linewidth}{
		     For large \(U\), Haldane and Jefferson find\footnote{{Haldane-1978, Jefferson-1977, Hewson, A. C.-1993-The Kondo Problem to Heavy Fermions}} three low energy theories:
\begin{itemize}
	\item the \textbf{frozen impururity fixed point} (\(\left<n_d\right> = 0\))
	\item the \textbf{local moment fixed point} (\(\left<n_d\right> = 1\)), and 
	\item the \textbf{valence fluctuation fixed point} (\(\left<n_d\right> \sim \frac{1}{2}\)).
% Not much is reported about their stability, and the method itself is perturbative, so it breaks down before reaching the low energy theory.  
\end{itemize}
    }
         \end{tabular}
           &   \begin{tabular}{c}
\includegraphics[scale=0.29]{haldane.png}
           \end{tabular}\\
   
\end{tabular}
}
\only<+>{
	\vspace*{-20pt}
\cen{\Large\textbf{NRG Results - Symmetric Model}}
\begin{tabular}{cl}  
\begin{tabular}{l}
             \parbox{0.5\linewidth}{
		     For the symmetric Anderson model\footnote{Krishna-murthy et al, 1980}:
		     \vspace*{10pt}
\begin{itemize}
\item the \textbf{free-orbital} fixed point (\(U=\Delta=0\)) - unstable
		     \vspace*{10pt}
     \item the \textbf{local moment} fixed point (\(U = \infty, \Delta=0\)) - saddle point, and 
		     \vspace*{10pt}
     \item the \textbf{strong-coupling} fixed point (\(\Delta=\infty, U = \text{finite}\)) - stable.
\end{itemize}
    }
         \end{tabular}
           &   \begin{tabular}{c}
\includegraphics[scale=0.35]{nrg.png}
           \end{tabular}\\
   
\end{tabular}
}
\only<+>{
	\vspace*{-100pt}
\cen{\Large\textbf{NRG Results - Asymmetric Model}}
Two more fixed points exist - 
	\vspace*{10pt}
\begin{itemize}
	\item the \textbf{valence fluctuation} fixed point (\(\epsilon_d = V = 0, U = \infty\)) 
	\vspace*{10pt}
\item the \textbf{frozen impurity} fixed point (\(U = V = 0, \epsilon_d = \infty\))
\end{itemize}
}
\end{frame}

\begin{frame}{Some Outstanding Questions}
  
\begin{itemize}
	\uncover<1->{\item Is it possible to get \textbf{nonperturbative scaling equations} for the whole journey?}
	\vspace*{10pt}
	\uncover<2->{\item Can we get Hamiltonians and wavefunctions in the \textbf{crossover region} (\(T \sim T_K\))?}
	\vspace*{10pt}
	\uncover<3->{\item What is the nature of the strong-coupling fixed point for a \textbf{finite system} (\(J \neq \infty\))? Will the results change if we introduce a \textbf{non-uniform density of states}?}
	\vspace*{10pt}
	\uncover<4->{\item Can we get \textbf{better estimates of dynamic quantities} in the crossover region?}
	\vspace*{10pt}
	\uncover<5->{\item Is it possible to show the \textbf{transfer of spectral weight} along the flow, possibly by tracking the bath spectral function or the many-particle entanglement?}
	\vspace*{10pt}
	\uncover<6->{\item How does NRG obtain the local moment in the \textbf{absence of hybridisation}?}
	\vspace*{10pt}
	\uncover<7->{\item Are there any interesting \textbf{topological aspects} of the fixed points?}
\end{itemize}

\end{frame}

%\begin{frame}{Motivation}
%  
%\begin{itemize}
%	\uncover<1->{\item "Poor man's" scaling\footnote{Haldane 1978, Jefferson 1977} is \textit{perturbative} and fails at large values - cannot show strong-coupling (SC) fixed point.}
%	\vspace*{20pt}
%	\uncover<2->{\item Instead, one needs to flow to large value of \(U\), do a Schrieffer -Wolff transformation and then flow to the SC fixed point. }
%	\vspace*{20pt}
%	\uncover<3->{\item \textbf{It would be nice to get a single set of equations that show the crossover to the strong-coupling fixed point.}}
%\end{itemize}
%
%\end{frame}
%
%\begin{frame}{Motivation}
%
%\begin{itemize}
%	\uncover<1->{\item Numerical Renormalization Group (NRG) does not provide any scaling equations - hard to figure out what is really happening.}
%	\vspace*{20pt}
%	\uncover<2->{\item NRG cannot show \textit{how the Hamiltonians and many-body wavefunctions vary along the flow} - projective in nature.}
%	\vspace*{20pt}
%	\uncover<3->{\item \textbf{It would be enlightening to see the flow into SC regime by tracking the change in entanglement} - hence we need wavefunctions.}
%\end{itemize}
%\end{frame}

\begin{frame}{Unitary Renormalization Group Formalism}

	\only<+>{\begin{minipage}{200pt}\cen{\textbf{The Short Version}}Apply \textit{unitary many-body transformations} to the Hamiltonian so as to successively \textit{decouple} high energy states and hence obtain scaling equations.\end{minipage}\begin{minipage}{250pt}\begin{figure}
\def\svgwidth{\columnwidth}
\centering
\scalebox{0.7}{\input{urg_short.pdf_tex}}
\end{figure}\end{minipage}}
\only<+>{\begin{minipage}{200pt}\cen{\large{\textbf{Step 1:}}}Start with the electrons farthest from the Fermi surface. Write the Hamiltonian as \textit{diagonal and off-diagonal terms} in this basis.\end{minipage}\begin{minipage}{300pt}
	\begin{figure}
\def\svgwidth{\columnwidth}
\centering
\input{matrix.pdf_tex}
\end{figure}
\end{minipage}}
\only<+>{\cen{\large{\textbf{Step 2:}}\\Rotate the Hamiltonian to kill the off-diagonal blocks.}
\begin{figure}
\def\svgwidth{\columnwidth}
\centering
\input{rotation.pdf_tex}
\end{figure}
% mention the commutator here
}
\only<+>{\cen{\large{\textbf{Step 3:}}\\Repeat the process with the new blocks.}
\begin{figure}
\def\svgwidth{\columnwidth}
\centering
\input{repeat.pdf_tex}
\end{figure}}
\only<+>{
\vspace{-50pt}
\cen{\Large{\textbf{Some Characteristic features of the URG}}}
\vspace{10pt}
\begin{itemize}
	\item Presence of the quantum fluctuation energy scale \(\omega\)
\vspace{10pt}
	\item Presence of finite-valued fixed points
\vspace{10pt}
	\item Spectrum-preserving transformations
\vspace{10pt}
	\item Tractable low-energy effective Hamiltonians
\end{itemize}
}
\end{frame}

\begin{frame}{Results}
\only<1>{
\large{\[\mathcal{H} = \overbrace{\sum_{k\sigma}\epsilon_k \hat n_{k\sigma} + \sum_{k\sigma}\left[V(k)c^\dagger_{k\sigma}c_{d\sigma} + \text{h.c.}\right] + \epsilon_d \sum_\sigma \hat n_{d\sigma} + U\hat n_{d\uparrow}\hat n_{d\downarrow}}^{SIAM} + \underbrace{J \vec{S_d}\cdot \sum_{kq\alpha\beta}\vec{\sigma}_{\alpha,\beta}c^\dagger_{k\alpha}c_{q\beta}}_\text{spin-spin interaction}\]}}
\only<2>{
	\vspace*{-15pt}
	\cen{\Large{\textbf{RG Equations}}
		\begin{equation*}\begin{aligned}\Delta U_n &= \left(U + \frac{1}{2}J\right)\sum_{|q|=\Lambda_n} \frac{|V(q)|^2}{(\omega - \epsilon_q - \frac{1}{2}U - \frac{1}{4}J)(\omega - \epsilon_q)}\\[5pt]
			\Delta V(q)_n &= -\frac{3}{4}J\sum_{|q|=\Lambda_n} \frac{V(q)}{\omega - \epsilon_q - \frac{1}{2}U - \frac{1}{4}J}\\[5pt]
\Delta J_n &= -\frac{1}{4}J^2\sum_{|q|=\Lambda_n\atop{k<\Lambda_n}} \frac{1}{\omega - \epsilon_q - \frac{1}{2}U - \frac{1}{4}J}\end{aligned}\end{equation*}
}
}
\only<3-5>{
	\cen{\Large{\textbf{The case of \(\pmb{J=0}\)}}
	\vspace*{5pt}
\begin{equation*}\Delta U_n = U\sum_{|q|=\Lambda_n} \frac{|V(q)|^2}{(\omega - \epsilon_q - \frac{1}{2}U)(\omega - \epsilon_q)}\end{equation*}}
	\vspace*{10pt}
\only<3>{\begin{figure}
\def\svgwidth{\columnwidth}
\centering
\scalebox{0.7}{\input{lmflow.pdf_tex}}
\end{figure}
}

\begin{tabular}{cl}  
\begin{tabular}{l}
	\hspace*{-20pt}\parbox{0.4\linewidth}{
	\begin{itemize}\uncover<4->{\item \textbf{No separatrix}\footnote{ Rukhsan, Vidhyadhiraja - Scaling analysis of the extended single impurity Anderson model} for the flows to the local moment.} 
		\uncover<5>{\item Local moment forms at finite \(U\).}
\end{itemize}
    }
         \end{tabular}
           &  \uncover<5>{ \begin{tabular}{c}
\includegraphics[scale=0.2]{UvsbareD.png}
	   \end{tabular}}\\
   
\end{tabular}

}
\end{frame}
\end{document}
