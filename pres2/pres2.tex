\documentclass[aspectratio=169]{beamer}
\usepackage{xcolor}
\definecolor{maroon}{rgb}{0.6,0,0}
\definecolor{bottlegreen}{rgb}{0,0.3,0}
\usepackage{lmodern,graphicx}
\usetheme{focus}
\newcommand{\cen}[1]{\begin{center}{#1}\end{center}}
%\usecolortheme{crane}
%\setbeamercolor{itemize item}{fg=darkred!80!black}

\title{Unitary Renormalization Group Approach to the Single-Impurity Anderson model}
\subtitle{}
\date{\today}
\author{Abhirup Mukherjee (18IP014)\\[5mm]{Supervisor: Dr. Siddhartha Lal}}
%\author[author2]

\institute{IISER Kolkata}

\begin{document}


\begin{frame}
\maketitle
\end{frame}

\begin{frame}{Outline}

\begin{itemize}
  \item The model
	  \vspace*{20pt}
  \item Motivation
	  \vspace*{20pt}
  \item Unitary Renormalization Group (URG) formalism
	  \vspace*{20pt}
  \item Results 
\end{itemize}

\end{frame}

\begin{frame}{The Single-Impurity Anderson Model}

%\vspace*{5pt}
%\begin{figure}
%\centering
%\def\svgwidth{\columnwidth}
%\centering
%\input{model.pdf_tex}
%\end{figure}
%\begin{figure}
%\centering
%\includegraphics[scale=0.55]{model_scheme.png}
%\end{figure}
	\scalebox{1.38}{
		\hspace*{-10pt}\(	\mathcal{H}_\text{siam} = \sum_{k\sigma}\epsilon_k \hat n_{k\sigma} + \sum_{k\sigma}\left[V(k)c^\dagger_{k\sigma}c_{d\sigma} + \text{h.c.}\right] + \epsilon_d \sum_\sigma \hat n_{d\sigma} + U\hat n_{d\uparrow}\hat n_{d\downarrow}
\)
}
\begin{figure}
\centering
\includegraphics[scale=0.55]{model_scheme.png}
\end{figure}

\end{frame}

\begin{frame}{Motivation}
  
\begin{itemize}
	\uncover<1->{\item "Poor man's" scaling\footnote{Haldane 1978, Jefferson 1977} is \textit{perturbative} and fails at large values - cannot show strong-coupling (SC) fixed point.}
	\vspace*{20pt}
	\uncover<2->{\item Instead, one needs to flow to large value of \(U\), do a Schrieffer -Wolff transformation and then flow to the SC fixed point. }
	\vspace*{20pt}
	\uncover<3->{\item \textbf{It would be nice to get a single set of equations that show the crossover to the strong-coupling fixed point.}}
\end{itemize}

\end{frame}


\begin{frame}{Motivation}

\begin{itemize}
	\uncover<1->{\item Numerical Renormalization Group (NRG) does not provide any scaling equations - hard to figure out what is really happening.}
	\vspace*{20pt}
	\uncover<2->{\item NRG cannot show \textit{how the Hamiltonians and many-body wavefunctions vary along the flow} - projective in nature.}
	\vspace*{20pt}
	\uncover<3->{\item \textbf{It would be enlightening to see the flow into SC regime by tracking the change in entanglement} - hence we need wavefunctions.}
\end{itemize}
\end{frame}

\begin{frame}{Unitary Renormalization Group Formalism}

	\only<+>{\begin{minipage}{200pt}\cen{\textbf{The Short Version}}Apply \textit{unitary many-body transformations} to the Hamiltonian so as to successively \textit{decouple} high energy states and hence obtain scaling equations.\end{minipage}\begin{minipage}{250pt}\begin{figure}
\def\svgwidth{\columnwidth}
\centering
\scalebox{0.7}{\input{urg_short.pdf_tex}}
\end{figure}\end{minipage}}
\only<+>{\begin{minipage}{200pt}\cen{\large{\textbf{Step 1:}}}Start with the electrons farthest from the Fermi surface. Write the Hamiltonian as \textit{diagonal and off-diagonal terms} in this basis.\end{minipage}\begin{minipage}{300pt}
	\begin{figure}
\def\svgwidth{\columnwidth}
\centering
\input{matrix.pdf_tex}
\end{figure}
\end{minipage}}
\only<+>{\cen{\large{\textbf{Step 2:}}\\Rotate the Hamiltonian to kill the off-diagonal blocks.}
\begin{figure}
\def\svgwidth{\columnwidth}
\centering
\input{rotation.pdf_tex}
\end{figure}}
\only<+>{\cen{\large{\textbf{Step 3:}}\\Repeat the process with the new blocks.}
\begin{figure}
\def\svgwidth{\columnwidth}
\centering
\input{repeat.pdf_tex}
\end{figure}}
\end{frame}

\begin{frame}{Results}
	\Large{\[\mathcal{H} = \sum_{k\sigma}\epsilon_k \hat n_{k\sigma} + \sum_{k\sigma}\left[V(k)c^\dagger_{k\sigma}c_{d\sigma} + \text{h.c.}\right] + U\hat n_{d\uparrow}\hat n_{d\downarrow} + J \vec{S_d}\cdot \sum_{kq\alpha\beta}\vec{\sigma}_{\alpha,\beta}c^\dagger_{k\alpha}c_{q\beta}\]}
\end{frame}
\end{document}
