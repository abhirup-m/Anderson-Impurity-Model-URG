\documentclass{beamer}
\usepackage{xcolor}
\definecolor{maroon}{rgb}{0.6,0,0}
\definecolor{bottlegreen}{rgb}{0,0.3,0}
\usepackage{lmodern,graphicx}
\usepackage[scale=2]{ccicons}
\usetheme{Pittsburgh}
\usecolortheme{crane}
%\setbeamercolor{itemize item}{fg=darkred!80!black}

\title{Unitary Renormalization Group Approach to the Single-Impurity Anderson model}
\subtitle{}
\date{\today}
\author{Abhirup Mukherjee (18IP014)\\[5mm]{Supervisor: Dr. Siddhartha Lal}}
%\author[author2]

\institute{IISER Kolkata}

\begin{document}

\maketitle

\begin{frame}{Outline}

\begin{itemize}
  \item The model
  \item Motivation
  \item URG formalism
  \item Results 
\end{itemize}

\end{frame}

\begin{frame}{The Single-Impurity Anderson Model}

\vspace*{20pt}
\begin{figure}
\centering
\def\svgwidth{\columnwidth}
\hspace*{-20pt}\input{model.pdf_tex}
\end{figure}
\vspace*{-40pt}
\centering\includegraphics[scale=0.4]{model_scheme.png}
%\begin{equation*}
%	\mathcal{H} = \sum_{k\sigma}\epsilon_k \hat n_{k\sigma} + \sum_{k\sigma}\left[V(k)c^\dagger_{k\sigma}c_{d\sigma} + \text{h.c.}\right] + \epsilon_d \sum_\sigma \hat n_{d\sigma} + U\hat n_{d\uparrow}\hat n_{d\downarrow}
%\end{equation*}

\end{frame}

\begin{frame}{Motivation}
  
\begin{itemize}
	\item "Poor man's" scaling\footnote{Haldane 1978, Jefferson 1977} is \textit{perturbative} and fails at large values - cannot show strong-coupling (SC) fixed point.
	\item Instead, one needs to flow to large value of \(U\), do a Schrieffer -Wolff transformation and then flow to the SC fixed point. 
	\item \textbf{It would be nice to get a single set of equations that show the crossover to the strong-coupling fixed point.}
\end{itemize}

\end{frame}


\begin{frame}{Motivation}

\begin{itemize}
	\item Numerical Renormalization Group (NRG) does not provide any scaling equations - hard to figure out what is really happening.
	\item NRG cannot show \textit{how the Hamiltonians and many-body wavefunctions vary along the flow} - projective in nature.
	\item \textbf{It would be enlightening to see the flow into SC regime by tracking the change in entanglement} - hence we need wavefunctions.
\end{itemize}

\end{frame}

\end{document}
